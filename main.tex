\documentclass{beamer}
\usepackage[slovene]{babel}
\usetheme{Madrid}
\usepackage{graphicx} % Required for inserting images
\usepackage{tikz}
\usepackage{wrapfig}
\usepackage{blindtext}


\title{NRO DN1}
\subtitle{Matlab, Beamer in GitHub}
\author{Martin Kavčič}
\institute{Univerza v Ljubljani Fakulteta za strojništvo}
\date{Oktober 2023}
\titlegraphic{\includegraphics[width=2cm]{FS-logo.jpg}}

\begin{document}

\begin{frame}
    \titlepage
\end{frame}

\begin{frame}{Kazalo}
    \tableofcontents
\end{frame}

\section{Matlab}
\section{Beamer}
\section{GitHub}

\begin{frame}{Matlab}
   
\begin{itemize}
    \item Postopek računanja $\pi$
    \pause
\begin{itemize}
    \item definicija funkcije mcc-pi.m
    \pause
    \item definicija fukcije calc-pi.m
    \pause
    \item vizualizacija podatkov
\end{itemize}
    
\end{itemize}
\end{frame}

\begin{frame}{Beamer}
 Okolje Beamer smo uporabili za izdelavo te kratke in skromne prve predstavitve.    
\end{frame}

\begin{frame}{GitHub}
\begin{columns}
% Column 1
\begin{column}{0.5\textwidth}
        Nato pa smo vse skupaj še naložili na naš reporzitorji Napredna računalniška orodja na GitHub-u.
\end{column}
% Column 2    
\begin{column}{0.5\textwidth}
    \begin{figure}
    \centering
        \includegraphics[width=0.7\textwidth]{github.png}
        \caption{GitHub logo}
    \end{figure}
\end{column}
\end{columns}
\end{frame}



\end{document}
